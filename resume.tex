%!TEX TS-program = xelatex
%!TEX encoding = UTF-8 Unicode
% Awesome CV LaTeX Template for CV/Resume
%
% This template has been downloaded from:
% https://github.com/posquit0/Awesome-CV
%
% Author:
% Claud D. Park <posquit0.bj@gmail.com>
% http://www.posquit0.com
%
% Template license:
% CC BY-SA 4.0 (https://creativecommons.org/licenses/by-sa/4.0/)
%
% Modifications have been made to the original.
%
% Author:
% Robert Hamrick <rph@muxux.org>
% https://github.com/hamr/resume


%-------------------------------------------------------------------------------
% CONFIGURATIONS
%-------------------------------------------------------------------------------
% A4 paper size by default, use 'letterpaper' for US letter
\documentclass[12pt, letterpaper]{awesome-cv}
\usepackage[markifdraft]{gitinfo2}

% Configure page margins with geometry
\geometry{left=1.4cm, top=.8cm, right=1.4cm, bottom=1.8cm, footskip=.5cm}

% Specify the location of the included fonts
% \fontdir[fonts/]

% Color for highlights
% Awesome Colors: awesome-emerald, awesome-skyblue, awesome-red, awesome-pink, awesome-orange
%                 awesome-nephritis, awesome-concrete, awesome-darknight
\colorlet{awesome}{awesome-red}
% Uncomment if you would like to specify your own color
% \definecolor{awesome}{HTML}{CA63A8}

% Colors for text
% Uncomment if you would like to specify your own color
% \definecolor{darktext}{HTML}{414141}
% \definecolor{text}{HTML}{333333}
% \definecolor{graytext}{HTML}{5D5D5D}
% \definecolor{lighttext}{HTML}{999999}

% Set false if you don't want to highlight section with awesome color
% \setbool{acvSectionColorHighlight}{true}
% \setbool{acvSectionColorHighlight}{false}

% If you would like to change the social information separator from a pipe (|) to something else
\renewcommand{\acvHeaderSocialSep}{\quad\textbar\quad}


%-------------------------------------------------------------------------------
%	PERSONAL INFORMATION
%	Comment any of the lines below if they are not required
%-------------------------------------------------------------------------------
% Available options: circle|rectangle,edge/noedge,left/right
% \photo[rectangle,edge,right]{./examples/profile}
\name{Robert}{Hamrick}
\position{Systems Engineering{\enskip\cdotp\enskip}Development{\enskip\cdotp\enskip}Operations}
% \address{6818 Snowdon Ave, El Cerrito, CA 94609}
\email{rph@muxux.org}
% \homepage{www.posquit0.com}
\github{hamr}
\linkedin{robert-hamrick-902939207}
% \gitlab{gitlab-id}
% \stackoverflow{SO-id}{SO-name}
% \twitter{@twit}
% \skype{skype-id}
% \reddit{reddit-id}
% \medium{madium-id}
% \googlescholar{googlescholar-id}{name-to-display}
%% \firstname and \lastname will be used
% \googlescholar{googlescholar-id}{}
\extrainfo{
  \href{https://github.com/hamr/resume/}{\faCodeBranch \texttt{src}} \faArrowRight~
  \href{https://github.com/hamr/resume/releases/latest}{\faFile* \textsc{latest}}
}

% \quote{``Be the change that you want to see in the world."}

%-------------------------------------------------------------------------------
\begin{document}

% Print the header with above personal informations
% Give optional argument to change alignment(C: center, L: left, R: right)
\makecvheader[C]

% Print the footer with 3 arguments(<left>, <center>, <right>)
% Leave any of these blank if they are not needed
\makecvfooter
  {\today}
  {Robert Hamrick~~~·~~~Résumé}
  {\texttt{\gitAbbrevHash\gitDirty}}


%-------------------------------------------------------------------------------
%	CV/RESUME CONTENT
%	Each section is imported separately, open each file in turn to modify content
%-------------------------------------------------------------------------------
\cvsection{Summary}

\begin{cvparagraph}

  Now working on development and operations for Pixar's datacenter-hosted workstation infrastructure.

  % \begin{cvbox}
  I'm a longtime enthusiast for the Linux desktop who enjoys working in a dynamic, highly collaborative environment like an animation studio.
  I like to engage with a diverse group of talented individuals---to solve the challenging, sometimes unexpected problems that emerge from creative tensions between artists and engineers.
  And I'm always looking to learn new technologies along the way.
  % \end{cvbox}

  Current responsibilities include:
  \begin{cvitems1}
    \item compose and test RHEL-based runtime for in-house and third party production applications
    \item build images, provision systems, and manage configuration for workstation and renderfarm nodes
    \item support Teradici remote desktops
    \item troubleshoot OS-level problems with other groups, e.g.\ Tools QA, Tools Build, Farm Ops, Desktop Support
    \item \emph{Help, I don't know who else to ask!}
  \end{cvitems1}

  Tools I use every day:
  \begin{itemize*}[]
    \item Python
    \item Ansible
    \item Systemd
    \item Docker
    \item Git
    \item Rpm
    \item Emacs
    \item Bash
    \item Qemu-Kvm
    \item Teradici
    \item clear, respectful prose.
  \end{itemize*}

\end{cvparagraph}

\cvsection{Work Experience}

\begin{cventries}

  \cventry
  {Systems Engineer, Senior}  % Job title
  {Pixar Animation Studios}  % Organization
  {Emeryville/Remote}  % Location
  {2021--Now}  % Date(s)
  {
    \begin{cvitems}  % Description(s) of tasks/responsibilities
    %% \item {}
    \end{cvitems}
  }

  \cventry
  {Systems Administrator, Senior}  % Job title
  {Pixar Animation Studios}  % Organization
  {Emeryville/Remote}  % Location
  {2018--2021}  % Date(s)
  {
    \begin{cvitems}  % Description(s) of tasks/responsibilities
    \item {Wrote tooling for composing, testing, and deploying releases of RHEL-based OS runtime.}
    \item {Updated Linux workstation platform on regular schedule (\textasciitilde{}2200 hosts, 2x/month).}
    \end{cvitems}
  }

  \cventry
  {Systems Administrator}
  {Pixar Animation Studios}
  {Emeryville}
  {2014--2017}
  {
    \begin{cvitems}
    \item {Collaborated on design and implementation of virtualized, Teradici-connected Linux workstations.}
    \item {Worked with artists to solve usability issues during migration to remote platform.}
    \item {Ported in-house configuration management to Ansible}
    \end{cvitems}
  }

  \cventry
  {Production Support Engineer, Lead}
  {Pixar Animation Studios}
  {Emeryville}
  {2012--2014}
  {
    \begin{cvitems}
    \item {Advocated for production artists having problems with applications or systems infrastructure.}
    \item {Instituted queue reviews to encourage task completion and balance engineer workloads.}
    \item {Scripted common workarounds and automated configuration to reduce labor required for common tasks.}
    \end{cvitems}
  }

  \cventry
  {Production Support Engineer}
  {Pixar Animation Studios}
  {Emeryville}
  {2008--2012}
  {
    \begin{cvitems}
    \item {Supported desktop Linux users, troubleshoot hardware, os, and application-level problems, escalating to other groups as needed.}
    \item {\textit{Keep animators animating.}}
    \end{cvitems}
  }

\end{cventries}

% \input{resume/honors.tex}
% \input{resume/presentation.tex}
% \input{resume/writing.tex}
% \input{resume/committees.tex}
\cvsection{Education}

%% \begin{cventries}

%% \cvparagraph{Academic writing samples available on request.}

\cventryshort
{MA in Linguistics} % Degree
{University of Chicago} % Institution
{Chicago} % Location
{1999} % Date(s)

\cventryshort
{BA in Anthropology}
{Tulane University}
{New Orleans}
{1994}

%---------------------------------------------------------
%% \end{cventries}

% \input{resume/extracurricular.tex}

\end{document}
